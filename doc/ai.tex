
\documentclass[a4paper]{article}
\usepackage{ucs}  % unicode
\usepackage[utf8x]{inputenc}
\usepackage[T2A]{fontenc}
\usepackage[bulgarian]{babel}
\usepackage{graphicx}
\usepackage{fancyhdr}
\usepackage{lastpage}
\usepackage{listings}
\usepackage{fancyvrb}
\usepackage[usenames,dvipsnames]{color}
\setlength{\headheight}{12.51453pt}

%%%%%%%%%%%%%% Pygments header.
\makeatletter
\def\PY@reset{\let\PY@it=\relax \let\PY@bf=\relax%
    \let\PY@ul=\relax \let\PY@tc=\relax%
    \let\PY@bc=\relax \let\PY@ff=\relax}
\def\PY@tok#1{\csname PY@tok@#1\endcsname}
\def\PY@toks#1+{\ifx\relax#1\empty\else%
    \PY@tok{#1}\expandafter\PY@toks\fi}
\def\PY@do#1{\PY@bc{\PY@tc{\PY@ul{%
    \PY@it{\PY@bf{\PY@ff{#1}}}}}}}
\def\PY#1#2{\PY@reset\PY@toks#1+\relax+\PY@do{#2}}

\def\PY@tok@gd{\def\PY@tc##1{\textcolor[rgb]{0.63,0.00,0.00}{##1}}}
\def\PY@tok@gu{\let\PY@bf=\textbf\def\PY@tc##1{\textcolor[rgb]{0.50,0.00,0.50}{##1}}}
\def\PY@tok@gt{\def\PY@tc##1{\textcolor[rgb]{0.00,0.25,0.82}{##1}}}
\def\PY@tok@gs{\let\PY@bf=\textbf}
\def\PY@tok@gr{\def\PY@tc##1{\textcolor[rgb]{1.00,0.00,0.00}{##1}}}
\def\PY@tok@cm{\let\PY@it=\textit\def\PY@tc##1{\textcolor[rgb]{0.25,0.50,0.50}{##1}}}
\def\PY@tok@vg{\def\PY@tc##1{\textcolor[rgb]{0.10,0.09,0.49}{##1}}}
\def\PY@tok@m{\def\PY@tc##1{\textcolor[rgb]{0.40,0.40,0.40}{##1}}}
\def\PY@tok@mh{\def\PY@tc##1{\textcolor[rgb]{0.40,0.40,0.40}{##1}}}
\def\PY@tok@go{\def\PY@tc##1{\textcolor[rgb]{0.50,0.50,0.50}{##1}}}
\def\PY@tok@ge{\let\PY@it=\textit}
\def\PY@tok@vc{\def\PY@tc##1{\textcolor[rgb]{0.10,0.09,0.49}{##1}}}
\def\PY@tok@il{\def\PY@tc##1{\textcolor[rgb]{0.40,0.40,0.40}{##1}}}
\def\PY@tok@cs{\let\PY@it=\textit\def\PY@tc##1{\textcolor[rgb]{0.25,0.50,0.50}{##1}}}
\def\PY@tok@cp{\def\PY@tc##1{\textcolor[rgb]{0.74,0.48,0.00}{##1}}}
\def\PY@tok@gi{\def\PY@tc##1{\textcolor[rgb]{0.00,0.63,0.00}{##1}}}
\def\PY@tok@gh{\let\PY@bf=\textbf\def\PY@tc##1{\textcolor[rgb]{0.00,0.00,0.50}{##1}}}
\def\PY@tok@ni{\let\PY@bf=\textbf\def\PY@tc##1{\textcolor[rgb]{0.60,0.60,0.60}{##1}}}
\def\PY@tok@nl{\def\PY@tc##1{\textcolor[rgb]{0.63,0.63,0.00}{##1}}}
\def\PY@tok@nn{\let\PY@bf=\textbf\def\PY@tc##1{\textcolor[rgb]{0.00,0.00,1.00}{##1}}}
\def\PY@tok@no{\def\PY@tc##1{\textcolor[rgb]{0.53,0.00,0.00}{##1}}}
\def\PY@tok@na{\def\PY@tc##1{\textcolor[rgb]{0.49,0.56,0.16}{##1}}}
\def\PY@tok@nb{\def\PY@tc##1{\textcolor[rgb]{0.00,0.50,0.00}{##1}}}
\def\PY@tok@nc{\let\PY@bf=\textbf\def\PY@tc##1{\textcolor[rgb]{0.00,0.00,1.00}{##1}}}
\def\PY@tok@nd{\def\PY@tc##1{\textcolor[rgb]{0.67,0.13,1.00}{##1}}}
\def\PY@tok@ne{\let\PY@bf=\textbf\def\PY@tc##1{\textcolor[rgb]{0.82,0.25,0.23}{##1}}}
\def\PY@tok@nf{\def\PY@tc##1{\textcolor[rgb]{0.00,0.00,1.00}{##1}}}
\def\PY@tok@si{\let\PY@bf=\textbf\def\PY@tc##1{\textcolor[rgb]{0.73,0.40,0.53}{##1}}}
\def\PY@tok@s2{\def\PY@tc##1{\textcolor[rgb]{0.73,0.13,0.13}{##1}}}
\def\PY@tok@vi{\def\PY@tc##1{\textcolor[rgb]{0.10,0.09,0.49}{##1}}}
\def\PY@tok@nt{\let\PY@bf=\textbf\def\PY@tc##1{\textcolor[rgb]{0.00,0.50,0.00}{##1}}}
\def\PY@tok@nv{\def\PY@tc##1{\textcolor[rgb]{0.10,0.09,0.49}{##1}}}
\def\PY@tok@s1{\def\PY@tc##1{\textcolor[rgb]{0.73,0.13,0.13}{##1}}}
\def\PY@tok@sh{\def\PY@tc##1{\textcolor[rgb]{0.73,0.13,0.13}{##1}}}
\def\PY@tok@sc{\def\PY@tc##1{\textcolor[rgb]{0.73,0.13,0.13}{##1}}}
\def\PY@tok@sx{\def\PY@tc##1{\textcolor[rgb]{0.00,0.50,0.00}{##1}}}
\def\PY@tok@bp{\def\PY@tc##1{\textcolor[rgb]{0.00,0.50,0.00}{##1}}}
\def\PY@tok@c1{\let\PY@it=\textit\def\PY@tc##1{\textcolor[rgb]{0.25,0.50,0.50}{##1}}}
\def\PY@tok@kc{\let\PY@bf=\textbf\def\PY@tc##1{\textcolor[rgb]{0.00,0.50,0.00}{##1}}}
\def\PY@tok@c{\let\PY@it=\textit\def\PY@tc##1{\textcolor[rgb]{0.25,0.50,0.50}{##1}}}
\def\PY@tok@mf{\def\PY@tc##1{\textcolor[rgb]{0.40,0.40,0.40}{##1}}}
\def\PY@tok@err{\def\PY@bc##1{\fcolorbox[rgb]{1.00,0.00,0.00}{1,1,1}{##1}}}
\def\PY@tok@kd{\let\PY@bf=\textbf\def\PY@tc##1{\textcolor[rgb]{0.00,0.50,0.00}{##1}}}
\def\PY@tok@ss{\def\PY@tc##1{\textcolor[rgb]{0.10,0.09,0.49}{##1}}}
\def\PY@tok@sr{\def\PY@tc##1{\textcolor[rgb]{0.73,0.40,0.53}{##1}}}
\def\PY@tok@mo{\def\PY@tc##1{\textcolor[rgb]{0.40,0.40,0.40}{##1}}}
\def\PY@tok@kn{\let\PY@bf=\textbf\dthod of a LatexFormatter returns a string containing \def commands ef\PY@tc##1{\textcolor[rgb]{0.00,0.50,0.00}{##1}}}
\def\PY@tok@mi{\def\PY@tc##1{\textcolor[rgb]{0.40,0.40,0.40}{##1}}}
\def\PY@tok@gp{\let\PY@bf=\textbf\def\PY@tc##1{\textcolor[rgb]{0.00,0.00,0.50}{##1}}}
\def\PY@tok@o{\def\PY@tc##1{\textcolor[rgb]{0.40,0.40,0.40}{##1}}}
\def\PY@tok@kr{\let\PY@bf=\textbf\def\PY@tc##1{\textcolor[rgb]{0.00,0.50,0.00}{##1}}}
\def\PY@tok@s{\def\PY@tc##1{\textcolor[rgb]{0.73,0.13,0.13}{##1}}}
\def\PY@tok@kp{\def\PY@tc##1{\textcolor[rgb]{0.00,0.50,0.00}{##1}}}
\def\PY@tok@w{\def\PY@tc##1{\textcolor[rgb]{0.73,0.73,0.73}{##1}}}
\def\PY@tok@kt{\def\PY@tc##1{\textcolor[rgb]{0.69,0.00,0.25}{##1}}}
\def\PY@tok@ow{\let\PY@bf=\textbf\def\PY@tc##1{\textcolor[rgb]{0.67,0.13,1.00}{##1}}}
\def\PY@tok@sb{\def\PY@tc##1{\textcolor[rgb]{0.73,0.13,0.13}{##1}}}
\def\PY@tok@k{\let\PY@bf=\textbf\def\PY@tc##1{\textcolor[rgb]{0.00,0.50,0.00}{##1}}}
\def\PY@tok@se{\let\PY@bf=\textbf\def\PY@tc##1{\textcolor[rgb]{0.73,0.40,0.13}{##1}}}
\def\PY@tok@sd{\let\PY@it=\textit\def\PY@tc##1{\textcolor[rgb]{0.73,0.13,0.13}{##1}}}

\def\PYZbs{\char`\\}
\def\PYZus{\char`\_}
\def\PYZob{\char`\{}
\def\PYZcb{\char`\}}
\def\PYZca{\char`\^}
% for compatibility with earlier versions
\def\PYZat{@}
\def\PYZlb{[}
\def\PYZrb{]}
%%%%%%%%%%%%%% Pygments header end.


\pagestyle{fancy}
%\fancyhead{}
\fancyfoot{}

\cfoot{\thepage\ от \pageref{LastPage}}

\addto\captionsbulgarian{%
  \def\abstractname{%
    Цел на проекта} %\cyr\CYRA\cyrs\cyrt\cyrr\cyra\cyrk\cyrt}}%
}

% Custom defines:
\def\js{\texttt{javascript}}
\def\jsg{JsGames}
\def\jsurl{http://iskren.info:50005/}

% TODO remove colorlinks before printing
\usepackage[unicode,colorlinks]{hyperref}   % this has to be the _last_ command in the preambule, or else - no work
\hypersetup{urlcolor=blue}
\hypersetup{citecolor=PineGreen}

 \begin{document}

\title{\jsg}
\author{
Зорница Атанасова Костадинова, 4 курс, КН, фн: 80227, ФМИ, \\
Искрен Ивов Чернев, 4 курс, КН, фн: 80246, ФМИ
}
\date{\today}
\maketitle

%\includegraphics[scale=0.1]{drop}

\begin{abstract}
Настоящият документ е курсова работа към проекта ``\jsg'' по предмета
``Изкуствен Интелект''. Описали сме задачата която си поставихме с този проект
и решението ѝ. Обяснени са алгоритмите по които агентите вземат решения за
следващ ход във всяка от игрите. Освен това сме представили програматичната
реализация на сайта - база данни, интерфейс, архитектура, използвани
технологии. Накрая говорим за възможностите за разширяване на проекта.

\end{abstract}
\newpage

\setcounter{tocdepth}{2}
\tableofcontents
\newpage

% TODO
% Мотивация, Задача на курсовата работа
% Кратък обзор
% Извършвана ли и някаква работа в тази област преди и ако да, каква? 
% Какви методи могат да се използват за решаването на задачата?
% Вашето решение
% Програмна реализация
% Резултати от експерименти (ако има)
% Заключение 
% Какво сте направили и какво още може да се направи
% Литература – цитирате всички използвани източници

\section{Описание на проекта}

Проект: cайт с javascript игри с изкуствен интелект.\footnote{Проектът се хоства на \jsurl.}

  \begin{itemize}
    \item сайта ще поддържа потребители със следната информация за тях:
    \begin{itemize}
      \item email
      \item парола
      \item име
    \end{itemize}

    \item информация за игра:
    \begin{itemize}
      \item име
      \item кратко описание
    \end{itemize} 
    
    \item за всяка изиграна игра ще се пази следната информация:
    \begin{itemize}
      \item коя е играта
      \item потребители играли играта (може някой да са компютри, в такъв случай се пази трудността на компютъра)
      \item резултат
      \item продължителност на играта
    \end{itemize} 

    \item възможност за изкарване на класиране:
    \begin{description}
      \item най-много изиграни игри \hfill \\ общо за всички игри или за конкретна игра
      \item най-много изкарани точки \hfill \\ общо точки или средно аритметично, общо и за конкретна игра 
      \item най-дълго прекарано време в игри \hfill \\ общо време или средно аритметично, общо или за конкретна игра
    \end{description} 

    \item ще има възможност за динамично добавяне, редактиране, изтриване на горепосочените данни където това има смисъл 

    \item допълнителни пояснения за игрите:
    \begin{itemize}
      \item игрите ще бъдат имплементирани на \js
      \item всички игри ще имат изкуствен интелект (имплементиран като \js\ клиент) с поне една степен на трудност
      \item сървърът ще поддържа комуникация между различни \js\ клиенти (браузъри) при игра на няколко души за да се обменят изиграните ходове
      \item ще може да се играе и само от един клиент (браузър) ако единия играч е човек, а другият компютър (изкуствен интелект)
    \end{itemize} 
  \end{itemize}

\section{Използвани технологии}

  \subsection{Haml}
  Haml е \cite{haml} маркъп език, с който може чисто и просто да се описва XHTML за всеки онлайн документ, без вграждане на код. Haml е предназначен като алтернатива на други шаблонни езици които вграждат код като PHP, ERB и ASP. Също така с Haml избягвате писането на чист XHTML с използването на семантична идентация.

  \subsection{Sass}
  Sass \cite{sass} е добавка към CSS3 \cite{css}, в която е добавено влагане на правила, променливи, модули, наследяване на селектори и други. Sass се трансформира до добре форматиран CSS с помощта на конзолна програма или плъгин към уеб фреймуърк.

  Sass има два синтаксиса. Новия синтаксис (въведен от Sass 3) още известен като SCSS е надмножество на CSS3. Това означава, че всеки валиден CSS3 файл е също валиден SCSS. SCSS файловете имат разширение \texttt{.scss}.

  Втория, по-стар синтаксис е известен като идентирания синтаксис (или просто ``Sass''). Вдъхновен от краткоста на Haml, той е предназначен за хора, който предпочитат изразителността пред близостта с CSS. Вместо точка и запетая и скоби се използва семантична идентация, за да се обособяват блокове. Въпреки че вече не е основния синтаксис, ще продължи да бъде поддържан. Файловете с идентиран синтаксис имат разширение \texttt{.sass}.

  \subsection{Ruby on Rails Framework}
  Ruby on Rails, често съкращаван като Rails или RoR, е уеб фреймуърк с отворен код за програмният език Ruby. Той е предназначен за програмиране по Agile методологията, която се използва от уеб програмисти за ускорено програмиране.

  Като много уеб фреймуърци, Ruby on Rails използва Model-View-Controller (MVC) архитектура за организиране на приложението.

  Ruby on Rails включва програмки, който правят често срещани задачи в програмирането по-лесни, като например автоматичното генериране на шаблонен код, за по-бързо стартиране на модел или изглед. Също включен е WEBrick - прост уеб сървър писан на Ruby, както и Rake - система за билдване (като make) която се дистрибутира като gem (пакет за Ruby). Заедно с Ruby on Rails тези програмки предоставят базова среда за програмиране.

  Ruby on Rails разчита на уеб сървър за да работи. Mongrel е предпочитан пред WEBrick, но също може да се ползва Lighttpd \cite{lighttpd}, Abyss \cite{abyss}, Apache \cite{apache}, nginx \cite{nginx} и много други. От 2008 уеб сървърът Passenger \cite{passenger} заменя Mongrel като препоръчителна опция.
  Ruby on Rails е известен също и с използването на \js\ библиотеките Prototype \cite{prototype} и Script.aculo.us за Ajax \cite{ajax}. В началото Ruby on Rails използва лек SOAP, но по-късно е заменен от REST. От версия 3 Ruby on Rails използва Unobtrusive \js\ \cite{unobtrusive} -- техника за разделяне на логиката от изгледа на уеб страницата.

  \subsection{NodeJs}
  Целта на Node.js \cite{node} е да предостави лесен начин за програмиране скалируеми, работещи в мрежа програми. Node инструктира операционната система да го уведомява за нови клиенти и после заспива. Ако някой се свърже се изпълнява предварително зададена функция (callback). За всяка връзка се използва съвсем малко памет.

  Това е в контраст с по-популярния модел на едновременна работа, в който се използват нишки. Мрежово програмиране имплементирано с нишки е сравнително неефективно и много трудно за използване. Node използва доста по-малко памет при високо натоварване, от колкото системи, които заделят по 2MB за всяка нишка / конекция. Още повече потребителите на Node няма нужда да се притесняват от dead-lock - просто защото няма заключване. Няма функции в Node, която директо изпълнява вход/изход - затова процеса никога не блокира. Поради този факт непрофесионалисти могат да пишат бързи системи.
    
  \subsection{jQuery}
  jQuery \cite{jquery} е \js\ библиотека предвидена да улесни скриптирането на HTML от страна на клиента. Тя е пусната през януари 2006 от Джон Ресиг. Използвана е в 41\% от 10000 най-посещавани сайтове. Това е най-популярната \js\ библиотека в момента.

  jQuery е безплатна, с отворен код, лицензирана под MIT и GNU v2 лицензите. Синтаксиса на jQuery е предвиден да улесни навигацията в документа, селектирането на DOM обекти, създаването на анимации, обработката на събития и разработването на Ajax приложения. jQuery също предоставя възможност за писане на плъгини. С използването на тези възможности програмиста може да създаде абстракции за анимация и взаимодействие от ниско ниво, сложни ефекти както и компоненти от високо ниво поддържащи теми. Това подпомага за създаването на сложни и динамични уеб страници.

  \subsection{Yui3 library}
  YUI \cite{yui} е \js\ библиотека разработвана от Yahoo!. Тя има добре развита модулна система, която позволява да се сваля само кода от библиотеката, който реално се използа. Системата ѝ за събити надгражда тази дефинирана в браузъра и позволява по-лесна обработка и добавяне на специализирани събития.

  \subsection{Mercurial}
  Mercurial \cite{mercurial} е дистрибутирана система за управление на сорс код.

  Традиционните системи за управление на сорс като Subversion са с типични клиент-сървър архитектури. Те имат централен сървър който пази информацията за различните ревизии на проекта. В контраст Mercurial е напълно дистрибутирана, позволявайки на всеки програмист да има локално копие на цялата история на проекта. Това означава, че програмистите са независими от връзката си с интернет или централния сървър. Добавянето, разклоняването и сливането в проекта са лесни и бързи операции.

  \subsection{RubyGems}
  RubyGems \cite{rubygems} е пакетен мениджър за програмният език Ruby, който предоставя стандартен формат за разпространението на Ruby програми и библиотеки във самодостатъчен файлов формат \texttt{.gem}. Той предоставя конзолен интерфейс за лесна инсталация и управление на гемове, както и сървър за тяхното разпространение. RubyGems е аналогичен на EasyInstall за програмният език Python. В момента RubyGems е част от стандартната библиотека на Ruby версия 1.9.

  \subsubsection{База от данни}
  PostgreSQL \cite{postgresql} е обектно-релационна система за управление на база от данни. Тя е лицензирана под MIT лиценз и следователно е безплатна и с отворен код. Както други системи с отворен код PostgreSQL не се контролира от нито една единствена компания -- интернационална група от програмисти и компании разработват продукта.

  \subsection{Статистика за проекта}
  \begin{itemize}
    \item Използвани езици за програмиране
    \begin{itemize}
      \item \js
      \item Ruby
      \item SQL
      \item html (генериран чрез haml)
      \item css (генериран чрез sass)
    \end{itemize}
    
    \item Редове код от началото на проекта - \texttt{5000}
    \begin{itemize}
      \item Ruby \quad 1200
      \item \js \quad 3200
    \end{itemize}

    \item Брой тестове
    \begin{itemize}
      \item Ruby \quad 21
      \item \js \quad 30
    \end{itemize}

    \item Брой commits в системата за контрол на версиите - \texttt{130}
  \end{itemize}

\section{Реализация}

\subsection{Програмен език}

Ruby \cite{ruby} and Javascript \cite{javascript}.
Ruby is a dynamic, open source programming language with a focus on simplicity and productivity. It has an elegant syntax that is natural to read and easy to write.

\subsubsection{Уеб сървъри}

  В проекта използваме два уеб сървъра:
  \begin{description}
    \item[WEBrick] \hfill \\ е стандартния web server за малки проекти при Ruby. Използваме го за да сервира страниците и статичното съдържание (картинки, javascript, css).
    \item[Node.js] \hfill \\ отговаря за комуникацията между различни клиенти посредством \texttt{socket.io}. Тази библиотека имплементира 6 различни протокола за двупосочна комуникация на \js\ клиент и произволен уеб сървър (стандартния ajax може да се използва само за поискване на информация - ако сървъра иска да каже нещо на клиента в произволно време трябва да се иползва нещо по-сложно). В момента имплементацията поддържа регистриране на потребителите в произволен брой канали, като всяко съобщение изпратено в канал се препраща на всички участници в канала без промяна. В бъдеще може да се имплементира цялата логиката на играта, за да има централизирано място, което да валидира ходовете и резултата преди да бъде изпратен към главния сървър.
  \end{description}

\subsubsection{База от данни}

Използваната база от данни в проекта е PostgreSQL \cite{postgresql}.
Схемата използва 4 таблици:

\begin{tabular} { | l | l | l | }
\multicolumn{3}{c}{Table "games"} \\
\hline
Column & Type & Modifiers \\
\hline
id & integer & not null default \\
name & character varying(255) & not null \\
display\_name & character varying(255) & \\
description & text & \\
\hline
\end{tabular}

\vspace{10 pt}

\begin{tabular} { | l | l | l | }
\multicolumn{3}{c}{Table "users"} \\
\hline
Column & Type & Modifiers \\
\hline
id & integer & not null default \\
game\_id & integer & \\
email & character varying(255) & not null default \\
encrypted\_password & character varying(128) & not null default \\
password\_salt & character varying(255) & not null default \\
name & character varying(255) & \\ 
\hline
\end{tabular}

\vspace{10 pt}

\begin{tabular} { | l | l | l | }
\multicolumn{3}{c}{Table "instances"} \\
\hline
Column & Type & Modifiers \\
\hline
id & integer & not null default \\
game\_id  & integer                     & \\
began    & timestamp without time zone & not null \\
duration & integer                     & default 0 \\
\hline
\end{tabular}

\begin{tabular} { | l | l | l | }
\multicolumn{3}{c}{Table "players"} \\
\hline
Column & Type & Modifiers \\
\hline
id & integer & not null default \\
instance\_id & integer & \\
user\_id     & integer & \\
play\_order  & integer & \\
score       & integer & default 0 \\
\hline
\end{tabular}

\subsection{Дизайн} 
Спазени са основните принципи на Обектно Ориентираното Програмиране \cite{oop} - капсулиране, полиморфизъм, наследяване, абстракция, обекти.

  \subsubsection{Дизайн на игрите}
  
    \begin{description}
      \item[JSG.GameCore.BaseGame] Това е базовият клас за всички игри. В него е имплементирана логиката са създаване на играчи. Подготвяне на информацията за реда на игра, спазване на реда на игра, съобщаването на резултата на сървъра. В повечето случай е досатъчно само да се имплементират методи за валидация на ход и детектор за край на играта. Този клас създава събития за това какъв ход е игран и кой играч е на ход за да може обектите играчи да са наясно с текущото състояние на играта.
      \item[JSG.GameCore.LocalUser] Представлява локално играещ потребител. Той трябва да слуша за събитията свързани с избиране на ход от дъската и да ги предава на обекта игра, който продължава тяхната обработка. Не е предвидено да се наследява.
      \item[JSG.GameCore.RemoteGateway] Свързващо звено при игри на отдалечени клиенти. Работата е да слуша за събития, касаещи потребители на отдалечени машини (например игра на локални потребители), и да ги препраща към отдалечените потребители посредством уникален канал за тази конкретна играта. Също слуша за действията на отдалечените играчи и съобщава на текущата игра.
      \item[JSG.Games.*.Game] Всяка игра трябва да имплементира този клас. Той е отговорен за дооформянето на базовия клас като имплементира методи за валидация и детектор за финални състояния.
      \item[JSG.Games.*.Board] Отговаря за логическото представяне на визуалната част на играта. Класът създава събития за всеки ход, който локалния потребител направи - например избор на поле в таблица. Също дава удобен интерфейс, за да може играта да отразява действията на отдалечените играчи без да се интересува от точната имплементация на визуалните елементи.
      \item[JSG.Games.*.BoardUI] Отговаря за визуалното представяне на играта. Този клас е от по-ниско ниво от Board -- методите му отговарят за пряка манипулация на DOM обекти, макар да представят по-удобен интерфейс, докато Board класа отговаря за отразяването и създаването на ходове. Например може кликването в няколко полета на дъската да създава един ход -- BoardUI класа изпраща събитие за всеки клик, докато в Board се решава кога група събития е ход.
      \item[JSG.Games.*.loader] Отговаря за зареждането на цялата игра динамично. Тъй като зареждането на всяка игра става от самата нея това намалява връзката между базовата библиотека и имплементатора на играта, което означава, че даже игри, изискващи по-сложно зареждане ще работи без да се променя базовия код на сайта.
      \item[JSG.GameCore.gameManager] Отговаря за подготовката преди стартирането на игра. Това включва взимане на детайлни данни относно играта и потребителите които участват. Също кода на играта се сваля динамично, в случай ве вече не е наличен. В случая с отдалечени потребители -- намира и свърза други потребители, който са изявили желание да играят същата игра. Когато необходимият брой потребители е налице се избира един активен, който регистрира играта на основния сървър и разпраща уникалния идентификатор на останалите потребители. Посредством този уникален идентификатор се осъществява връзка за обмен на ходове по време на играта.
    \end{description}

  \subsubsection{Дизайн на Потребителския интерфейс}
  \begin{description}
    \item[JSG.UI.mainUI] Отговаря за построяването на целия графичен интефейс. Това включва създаването на табулярния интерфейс и използването на останалите UI компоненти за да го попълни.
    \item[JSG.UI.GameList] Отговаря за показване на списък с игри. Този клас се използва както и в списъка за игра, така и в страницата със статистиките.
  \end{description}

  \begin{description}
    \item[JSG.Util] Този модул включва голямо число помощни функции. Те надграждат стандартната библиотека на \js\ и се използват във всички останали класове. Включват функции улесняващи обектно-ориентираното програмиране, работата с масиви и хешове, динамично сваляне на съдържание.
    \item[JSG.Util.Event] Този модул съдържа елементарна имплементация за създаване и слушане на събития. Използва се от почти всички модули, тъй като подобрява преизползването на код чрез намаляване на връзката между създаващия и слушащия събитието.
    \item[JSG.Util.HTML] Този модул съдържа група функции за лесно и бързо създаване на DOM обекти чрез \js. Чрез него динамичното създаване на HTML е бързо и лесно. Възможно е и слушането на събития произлизащи от DOM обектите.
  \end{description}

\section{Изкуствен интелект}
  \subsection{SG теория}
  SG теорията е мощен механизъм за решаване на игри, състоящи се от съвкупност
  от под-игри независими по между си.
  
    \subsubsection{Изисквания}
    Изисквания за игра, която искаме да решим чрез SG теория:
    \begin{itemize}
      \item игра $G$
      \item дваме играчи $A$ и $B$
      \item играта $G$ е съставена от под-игри $g_1, g_2, \dots,
      g_n$\footnote{допустимо е под-игрите да се създават и завършват в хода на
      играта}
      \item $g_i \quad i \in \overline{1,n}$  е игра за двама играчи
      \item всеки играч на ход има право да избере една под-игра $g_i \quad i \in
      \overline{1,n}$ и да направи ход в нея. След това другият играч е на ход,
      без да има каквито и да е ограничения за това в коя игра ще играе
      \item победител е играчът, победил в последната останала игра.
    \end{itemize}
  
    \subsubsection{Теория}
    Всяка игра протича като последователност от \em състояния\em. След хода на
    всеки играч играта преминава в друго състояние. В стандартната теория на
    игрите всяко състояние бива определено като:
    \begin{itemize}
      \item \em печелившо \em -- играчът на ход има печеливша стратегия \\
      \item \em губещо \em -- за всеки ход на текущия играч, другият има
      печеливша стратегия \\
      \item \em реми \em -- съществуват ходове само към губещи състояния
      и други реми състояния
    \end{itemize}
    Реми състоянията се различават от губещите по това, че ``следвайки'' реми
    състояния се достига до \em финално \em реми състояние -- т.е. такова
    състояние, което според правилата на играта е реми -- никой не печели,
    а губещо състояние ще доведе до финално губещо (другият печели по
    правилата на играта).

    \paragraph{sg стойности}
    При игрите разглеждани от SG  теорията няма реми състояния. При тях всяко
    състояние получава неотрицателна целочислена \em sg стойност\em. sg
    стойност $0$ означава губещо състояние, sg стойност $\neq 0$ означава
    печелившо състояние.

    Губещите състояния получават sg стойност $0$. Ако на дадено състояние $s$
    са известни sg стойностите на всичките му наследници (състоянията в които
    може да се стигне) $s_1, s_2, \dots, s_p$ -- съответно $sg[s_1], sg[s_2],
    \dots, sg[s_p]$, то sg стойността на $s$ се пресмята по формулата
    $$
      sg[s] = \mathrm{mex}(sg[s_1], sg[s_2], \dots, sg[s_p])
    $$
    където $mex$ -- \em{}minimal excludant \em е дефинирано като:
    $$
      \mathrm{mex}(a_1, \dots, a_n) = \min \{ 0, 1, 2, \dots \} \setminus \{ a_1, \dots, a_n \}
    $$

    От дефиницията на sg стойности лесно се вижда, че губещите състояния имат
    само ходове към печеливши, а печелившите състояния имат поне един ход към
    губещо. Това, което прави sg стойностите полезни е възможността за
    \em{}събиране\em на игри. От изискванията по-горе: игрите $g_1, g_2, \dots,
    g_n$ са сумирани за да се получи $G$. Ще записваме $g_1 \oplus g_2 \oplus
    \dots \oplus g_n = G$. Приятното свойство е, че $$sg[G] = sg[g_1] \oplus
    sg[g_2] \oplus \dots \oplus sg[g_n]$$ където $\oplus$ е операцията побитово
    XOR (изключващо или).

    \paragraph{доказателство на сумирането на игри}
    За да докажем горното твърдение е достатъчно да покажем, че ако дадени $n$
    числа имат XOR $s$ -- $a_1 \oplus a_2 \oplus \dots \oplus a_n = s$, то за
    всяко $0 \le p \le s$ съществува $i \in [1, n]$ и $0 \le a_i' \le a_i$,
    такива че $a_1 \oplus a_2 \oplus \dots \oplus a_i' \oplus \dots \oplus a_n
    = p$.

    За целта ще изразим $a_1 \oplus a_2 \oplus \dots a_{i-1} \oplus a_{i+1}
    \oplus \dots \oplus a_n = s \oplus a_1$. Сега заместваме във второто
    равенство и получаваме $a_i' \oplus s \oplus a_i = p$. От тук тривиално
    $a_i' = (p \oplus s) \oplus a_i$. Сега остава да докажем, че съществува $i$
    за което $a_i' < a_i$.

    Първо ще покажем, че при $p < s \quad p \oplus s$ има най-висок бит който
    е $1$ в $s$; с други думи най-високия бит на $p \oplus s$ идва от $s$.
    Понеже $p < s$ то в двоичен запис гледано от старши към младши битове $p$
    и $s$ съвпадат до един момент, след което се различават -- като
    задължително в най-старшия бит в който се различават $s$ е $1$ (а $p$
    e $0$, иначе $p > s$). Така в $p \oplus s$ старшите битове, в които $p$
    и $s$ съвпадат ще бъдат $0$, а първият бит в който се различават ще бъде от
    $s$.

    Сега остава да забележим, че за всеки бит $1$ в $s$ съществува $i$ за което
    $a_i$ има този бит $1$. Това е така, защото $s = a_1 \oplus a_2 \oplus
    \dots \oplus a_n$ и за всеки бит $1$ в $s$ има нечетен брой индекси, чиито
    $a$-та имат $1$ в този бит.

    \paragraph{оптимална стратегия}
    В случай, че $sg[G] \ne 0$ използвайки идеята от горното доказателство
    намираме под-игра $g_i$ в която може да направим ход, който да намали sg
    стойността ѝ по такъв начин, че $sg[G] \oplus sg[g_i] \oplus sg[g_i'] = 0$.
    Т.е. това ни позволява във всяка игра с ненулева sg стойност да изберем
    под-игра и да направим валиден ход в нея, така че новополученото състояние
    в $G'$ да има sg стойност $0$ -- губеща за опонента.

    \subsubsection{Nim}
    Nim е математическа стратегическа игра, произхождаща от дълбока древност.
    \paragraph{Правила}
    Състои се от $n$ купчинки в които има камъчета. На всеки ход играчът избира
    произволна купчинка и маха произволен брой камъчета от нея (може
    и всичките). Печели играчът взел последния камък.
    
    \paragraph{Решение}
    Всяка купчинка е независима от останалите. Освен това може лесно да бъдат
    пресметнати sg стойностите на купчинка с $p$ камъка -- $sg[h_p] = p$.
    Доказва се индуктивно. Сега по формулата може да пресметнем и sg стойността
    на цялата игра -- просто XOR-ваме броя камъни във всички купчинки -- нека
    резултатът е $t$. Намираме най-високия вдигнат бит в $t$ и купчинка $h_r$,
    чиито брой камъни $r$ има този бит вдигнат. От тук лесно $r \oplus t$
    е новия брой камъни които трябва да останат в купчината -- т.е. вадим $r
    - r \oplus t$ камъка.

    \subsubsection{Rocks}
    \paragraph{Правила}
    Игралната дъска се състои от редица камъни. Играчите се редуват, като всеки
    има право или да вземе един камък, или два последователни. Печели играчът, взел
    последния камък.

    \paragraph{Решение}
    При малък брой камъни играта може да бъде атакувана и с други
    по-конвенционални подходи. Тук ще илюстрираме решение чрез SG теорията.

    По време на играта камъните биват разбити на последователности от невзети
    камъни. Тъй като всеки играч може да вземе само веднъж и то последователни
    камъни, той ефективно играе само върху една от последователностите. След
    неговия ход може да се образуват нови последователности (ако вземе от
    средата), да се модифицират текущи (ако вземе от края) и да изчезнат (ако
    вземе всички камъни от една последователност -- трябва да са $1$ или $2$).
    Така естествено се образуват непресичащите се под-игри, чиито сума е цялата
    игра.

    \subparagraph{Пресмятане на sg стойностите на под-игрите}
    Една под-игра е дефинирана единствено от броя камъни -- те са винаги
    последователни, следователно структурата е една и съща. Най-лесният начин
    за пресмятането на sg стойностите е чрез обхождане на всички възможни
    ходове на всяка ситуация и пресмятането на sg стойността на новополучената
    позиция. Ако се намираме в позиция с $r$ камъка, то всеки ход би довел до
    позиция с по-малък общ брой камъни (независимо в колко последователности).
    Достатъчно е да сме пресметнали sg стойностите за редици с по-малко камъни.

    \subparagraph{Намиране на оптимален ход за всяко състояние}
    След като пресметнем sg стойността на състоянието -- XOR на sg стойностите
    на отделните последователности, трябва да открием последователността, чиито
    sg стойност съдържа най-високия бит на пълната sg стойност. След това
    е достатъчно да обходим всички възможни ходове -- $\Theta(n)$ по големината
    на последователността и да проверим кой от тях нулира sg стойността на
    цялата игра.

    \subsubsection{TicTacToe}
    Популярната игра ``морски шах''.

    \paragraph{Правила}
    Игралната дъска се състои от $3 \times 3$ дъска в която двамата играчи
    слагат алтернативно своя знак в свободна клетка. Печели първият играч,
    попълнил 3 квадратчета в една линия със своя знак.

    \paragraph{Решение}
    Тъй като броя на валидните позиции е само около 2000 пълно обхождане на
    всички ходове и маркиране на печелившите и губещи състояние върши работа.
    По-подробно -- всеки ход бива обходен в началото, за да бъде категоризиран
    като невалиден, финален с победител и финален реми. Освен това се
    преброяват валидните ходове които могат да бъдат направени от него. Всички
    финални ходове с победител се слагат в опашка. На следващата итерация се
    пуска BFS с опашка съдържаща финалните ходове. Всички достижими от тях
    стават печеливши -- първо ниво на BFS. За всички достижими от първото ниво
    бива намалена изходната им степен, преброена при първоначалното обхождане.
    Ако степента стане 0 тогава от съответното състояние има ходове само до
    печеливши състояния, следователно маркираме като губещо и слагаме
    в опашката. Формално погледнато състоянията на нечетни нива на BFS са
    печеливши, а състоянията на четни нива са губещи. Състоянията които никога
    не влизат в опашката са реми.

    По време на игра ако агентът е в печелившо състояние, той прави подходящия
    ход, за да докара потребителя в губещо състояние. Ако е в губещо състояние
    играе случайно, за да не се развиват еднообразни игри.

\section{Използвани програми}

За да разработим \jsg \ си инсталирахме:
\begin{itemize}
  \item[mercurial] Изтегля се от официалния сайт или през пакетен мениджър. Чрез него се изтегля сорса на проекта от \texttt{http://iskren.info:3001}.
  \item[PostgreSQL] Изтегля се от официалния сайт или през пакетен мениджър. \\
    След инсталацията трябва да се създаде потребител с пълни права както и 3 бази, който му принадлежат.
  \item[jsl] Изтегля се от официалния сайт и се настройва текстовия редактор да го използва при запазване на \js\ файлове. Показва възможни грешки за да помогне на по-ранното им отстраняване.
  \item[jquery] Изтегля се от официалния сайт.
  \item[jquery.log] Плъгин за jquery за печатане на съобщения в дебъг конзолите.
  \item[qunit] Плъгин за jquery за писане на unit test-ове.
  \item[ruby] Програмният език - от официалния сайт или през пакетен мениджър.
  \item[ruby:minitar] Плъгин за поддръжка на \texttt{tar} архиви от Ruby. Инсталира се с \texttt{gem} след като е свален от официалния сайт.
  \item[ruby:rails] Може да се инсталира с \texttt{gem}.
  \item[rails:jquery] Адаптер на rails.js използващ jQuery. Това е необходимо, тъй като Prototype и jQuery не могат да бъдат използвани едновременно. Инсталира се от официалния сайт \footnote{https://github.com/rails/jquery-ujs}.
  \item[rails:devise] Плъгин за потребители в rails. Инсталира се с \texttt{gem}.
  \item[rails:haml] Плъгин за haml и sass. Инсталира се с \texttt{gem}.
  \item[rails:foreigner] Плъгин за добавяне на foreign key в базата данни чрез миграция. Инсталира се с \texttt{gem}.
  \item[node] От официални сайт се сваля и компилира. Препоръчително е да се инсталира в директория, в която user-а има права.
  \item[node:npm] От официалния сайт се сваля и инсталира. Чрез него се инсталират всички плъгини за Node.
  \item[node:socket.io] Плъгин за Node. Инсталира се през npm.
  \item[node:json] Плъгин за Node. Инсталира се през npm.
  \item[node:yui3] Плъгин за Node. Инсталира се през npm.
\end{itemize}

\newpage

\section{Декларация за липса на плагиатство}
Тази курсова работа е наше дело, като всички изречения, илюстрации и програми
от други хора са изрично цитирани.  Тази курсова работа или нейна версия не са
представени в друг университет или друга учебна институция.  Разбираме, че ако
се установи плагиатство в работата ни ще получим оценка “Слаб”.

\vspace{2cm}
Зорница Атанасова Костадинова

\vspace{2cm}
Искрен Ивов Чернев

\newpage

\begin{thebibliography}{99}
  \bibitem{haml} \url{http://haml-lang.com/}
  \bibitem{sass} \url{http://sass-lang.com/}
  \bibitem{css} \url{http://en.wikipedia.org/wiki/Css}
  \bibitem{webServer} \url{http://en.wikipedia.org/wiki/Web\_server}
  \bibitem{lighttpd} \url{http://www.lighttpd.net/}
  \bibitem{abyss} \url{http://www.aprelium.com/}
  \bibitem{apache} \url{http://www.apache.org/}
  \bibitem{nginx} \url{http://nginx.org/}
  \bibitem{passenger} \url{http://www.modrails.com/}
  \bibitem{prototype} \url{http://www.prototypejs.org/}
  \bibitem{ajax} \url{http://en.wikipedia.org/wiki/Ajax\_(programming)}
  \bibitem{unobtrusive} \url{http://en.wikipedia.org/wiki/Unobtrusive\_JavaScript}
  \bibitem{jquery} \url{http://jquery.com}
  \bibitem{yui} \url{http://developer.yahoo.com/yui/}
  \bibitem{mercurial} \url{http://mercurial.selenic.com/}
  \bibitem{rubygems} \url{http://rubygems.org/}
  \bibitem{ruby} \url{http://www.ruby-lang.org/en/}
  \bibitem{javascript} \url{http://en.wikipedia.org/wiki/ECMAScript}
  \bibitem{webrick} \url{http://en.wikipedia.org/wiki/WEBrick}
  \bibitem{node} \url{http://nodejs.org/}
  \bibitem{postgresql} \url{http://www.postgresql.org/}
  \bibitem{oop} \url{http://en.wikipedia.org/wiki/Object-oriented\_programming}
\end{thebibliography}

\end{document}
